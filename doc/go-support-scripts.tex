\documentclass[a4paper]{article}
\usepackage{url}
\usepackage[a4paper, total={6.5in, 10in}]{geometry}
\usepackage[utf8]{inputenc}
\usepackage[english]{babel}
\usepackage[]{longtable}
\usepackage[
backend=biber,
style=alphabetic,
]{biblatex}
\usepackage[]{csquotes}
\usepackage[]{ukdate}
\date{\today{} --- Version \versionNr{}}

\addbibresource{latexSkeleton.bib}
\setlength\parindent{0pt}
\addtolength{\parskip}{5pt}

\begin{document}
\newcommand{\versionNr}{0.0.1}


\title{go Support Scripts}
\author{Christian ENGEL}
\maketitle


\tableofcontents

\section{About}

The use of go (alias golang) is a pretty succesful one.  The language has been used successfully to development
REST-API services and front-ends. The advantages of this language and in particular the implementation are:

\begin{itemize}
    \item no virtual CLR or JRE runtime, so comparably small and fast
    \item every installation contains all components for cross-compilation
    \item sucessfully compiled the same application for ARM5, ARM7, ARM64, and AMD64 environments
    \item libraries looks like designed from programming professionals, not lecturers
    \item good, helpful community existing
\end{itemize}

\section{Directory Design}

\section{Helper Scripts}

The scripts are all part of the \texttt{ConfigShell/bin} directory. They can be classified like:

\begin{enumerate}
    \item compilation scripts
    \item version-number increase and compilation scripts
    \item execution scripts to execute the last compiled binary
    \item status check scripts
    \item cleanup scripts
\end{enumerate}

The commands are grouped like:

\fbox{%
    \begin{minipage}[h]{.9\textwidth}
        \begin{center}
            \begin{tabular}{l|l|l|l}
            \textbf{bumpversion} \& & \textbf{build}& \textbf{execute} & \textbf{other} \\
            \textbf{debug-build} && \textbf{last binary}& \\
            \hline
            go-major, goma & godebug & goexec-debug, goed & go-status, gost \\
            go-minor, gomi &&& godistclean \\
            go-patch, gopa &&&\\
            & gorelease & goexec-release, goer& \\
            & goupx & goexec-upx, goeu & 
            \end{tabular}
        \end{center}
    \end{minipage}
}

\end{document}