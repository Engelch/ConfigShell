\documentclass[a4paper]{article}
\usepackage{url}
\usepackage[utf8]{inputenc}
\usepackage[english]{babel}
\usepackage[
backend=biber,
style=alphabetic,
]{biblatex}
\usepackage[]{csquotes}
\usepackage[]{ukdate}

\addbibresource{latexSkeleton.bib}
\setlength\parindent{0pt}
\addtolength{\parskip}{5pt}

\begin{document}
\title{title}
\author{CIAS}
\maketitle

\tableofcontents

\begin{itemize}
    \item Author: Fantomas
    \item Version[/last update] of this document:  0.0.0 / 2023-01
    \item Created: 2023-01
    \item Version of described solution:
\end{itemize}

\section{About}



\texttt{pandoc}~\cite{pandoc_homepage} is to be installed on your development machines using the related \texttt{ansible}~\cite{ansible_homepage} *playbooks*.

\medskip

\noindent
The above text was created using
\begin{verbatim}
    \texttt{pandoc}~\cite{pandoc_homepage} is to be installed on
    your development machines using the related
    \texttt{ansible}~\cite{ansible_homepage} *playbooks*.
\end{verbatim}

\noindent
Please look at the example to learn more about \LaTeX{} commands. Here, some fancy math:

This is a simple math expression with numbering

\begin{equation}
    \sqrt{x^2+1}    \label{eq0}
\end{equation}

separated from text. We can refer to it using the label \texttt{eq0} which we defined using the \verb?\label{eq0}? statement (this is equation~\ref{eq0} and the page~\pageref{eq0}).
The same kind of labeling can be used for sections, subsection,..., tables, images.


\section{License}
....
\appendix

\section{Special Characters}

äöüß can be typed as \verb?äöüß?.

\section{Creation of PDF documentation}

The documentation can be created in a few steps:

\fbox{%
\begin{minipage}{0.92 \textwidth}
    \begin{enumerate}
        \item latex latexSkeleton
        \item biber latexSkeleton
        \item xelatex latexSeketon
        \item xelatex latexSeketon
    \end{enumerate}
\end{minipage}
}

The first \texttt{latex} call tries to create a dvi file and additionally creates a \texttt{.aux} file which is used by biber to
extract the required citations from the \texttt{latexSkeleton.bib} file.

The \texttt{biber} command reads the \texttt{.aux} file and creates a \texttt{.bbl file} with the extracted bibliographic references.

The last steps compile the \LaTeX{} document again. As we are interested in PDF output and want to have support for UTF-8 character encoding we used \texttt{xelatex} here.
Alternatively, we could have used the \texttt{latex} command --- until no comments tell us to rerun it again --- and then we would have used \texttt{dvipdf} to convert the dvi
into a PDF file.

\appendix
\printbibliography
\end{document}